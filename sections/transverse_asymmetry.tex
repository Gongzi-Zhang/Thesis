\chapter{Transverse Asymmetry}
The beam normal single spin asymmetry (BNSSA), also known as the transverse single spin asymmetry
or transverse asymmetry, is distinct from the PV asymmetry. It is purely EM 
and, therefore, preserves parity. The BNSSA arises from the interference
between the one-photon and two-photon exchanges (OPE and TPE), making it sensitive 
to the TPE amplitude. By measuring the BNSSA, we can probe the strength of the TPE, which
plays a crucial role in electron elastic scattering that may explain the discrepancy
in the proton radius measurements obtained through different methods.

The transverse asymmetry is also a significant systematic uncertainty in PV 
asymmetry measurements due to residual transverse polarizations in the electron beam. 
With $\CA_n \sim \alpha_{EM}m_e/E_e$, its magnitude of $10^{-5}$ (10~ppm)
for a GeV-level electron beam is much larger than $\CA_{\text{PV}}$. Therefore, 
a thorough understanding and precise measurement of the transverse asymmetry is necessary
to ensure the accuracy of $\CA_{\text{PV}}$.

Being a routine and bonus of a PV experiment, PREX-I also measured the transverse
asymmetry of several nuclei, namely ${}^{1}$H, \He, \Carbon and \Pb. Surprisingly, PREX-I
saw a zero transverse asymmetry in \Pb, while the transverse asymmetries of other 
light nuclei agreed with theoretical predictions, as shown in 
Fig.~\ref{fig:PREX-I_AT}. This discrepancy in \Pb remains a puzzle and is one of the motivations for conducting PREX-II. 
\begin{figure}[!h]
    \centering
    \includegraphics[width=0.5\linewidth]{PREX-I_AT}
    \caption{Transverse asymmetries measured in PREX-I \cite{PhysRevLett.109.192501}.}
    \label{fig:PREX-I_AT}
\end{figure}

As its name implies, the BNSSA depends on the polarization of either the target or the
electron beam, but not both simultaneously. In this regard, a polarized electron 
beam is prefered over a polarized target,
because it is hard to polarize nuclei, especially heavy nuclei.

%%%%%%%%%%%%%%%%%%%%%%%%%%%%%%%%%%%%%%%%%%%%%%%%%%%%%%%%%%%%%%%%%%%%%%%%
\section{Motivation for the Transverse Asymmetry}

%%%%%%%%%%%%%%%%%%%%%%%%
\subsubsection{The Scattering Theory}
Consider the scattering of a free particle ($t_0 \rightarrow -\infty$) initially
in a state $\ket{i}$ from a time-independent potential $V(\vec{r})$, which 
decays quickly as $r \rightarrow \infty$. 
In the Schr\"odinger picture ($\hbar = 1$), the time evolution of the system is
represented by the state $\ket{\psi(t)}_S$ 
\begin{equation}
    \ket{\psi(t)}_S = U(t)\ket{\psi(t_0)} = \lim_{t_0 \rightarrow -\infty}U(t, t_0)\ket{i}
\end{equation}
where $U(t, t_0)$ is the evolution operator:
\begin{equation}
    U(t, t_0) = \exp(\frac{1}{i}(H_0 + V)(t - t_0)) = \exp(-i(H_0 + V)(t-t_0))
\end{equation}
$H_0$ is the free Hamiltonian and $H = H_0 + V$ is the complete Hamiltonian
with the interaction term. 

The projection of $\psi(t)$ to a free final state $\ket{f}$ defines the so-called
S-matrix (the order of the subscripts matters):
\begin{equation}
    S_{if} \equiv \lim_{t\rightarrow +\infty}\bra{f}\ket{\psi(t)} 
    = \lim_{t\rightarrow\infty} \lim_{t_0 \rightarrow -\infty} \bra{f} U(t, t_0) \ket{i}
\end{equation}
which defines the S operator:
\begin{equation}
    S_{if} = \bra{f} S \ket{i} \Longrightarrow S = U(+\infty, -\infty)
\end{equation}

The S-matrix describes the scattering amplitude from a free initial state $\ket{i}$
to a free final state $\ket{f}$. Conservation of the probability indicates
unitary of the S matrix:
\begin{equation}
    S^\dag S = \sum_f |\bra{f} U(+\infty, -\infty) \ket{i}|^2 = 1
\end{equation}

It is easier to evaluate $U(t)$ in the interaction picture. Define
\begin{equation}
    \ket{\psi(t)}_I \equiv \exp(-\frac{1}{i}H_0 t) \ket{\psi(t)}_S 
    = \exp(iH_0 t) \exp(-i (H_0 + V) t)\ket{i}
\end{equation}
The subscript I and S denote the interaction and Schr\"odinger picture respectively.
The evolution of $\ket{\psi(t)}_I$ is:
\begin{equation}
    \begin{aligned}
	\frac{d}{dt}\ket{\psi(t)}_I 
	&= \left[ \exp(i H_0 t) (iH_0) \exp(-i (H_0 + V) t)
	+ \exp(i H_0 t) (-i)(H_0 + V) \exp(-i (H_0 + V) t) \right] \ \ket{i}  \\
	&= -i\ \exp(i H_0 t)\ V \ \exp(-i (H_0 + V) t) \ket{i}   \\
	&= -i \exp(iH_0 t) V \exp(-iH_0 t) \cdot \exp(iH_0 t) \exp(-i(H_0 + V)t) \ket{i}   \\
	&= -i V_I(t) \ket{\psi(t)}_I
    \end{aligned}
    \label{eq:interaction_evolution}
\end{equation}
where $V_I(t) = \exp(iH_0t)V\exp(-iH_0t)$ is the time dependent interaction term.
Eq.~\ref{eq:interaction_evolution} leads to the Dyson series:
\begin{equation}
    U(t, t_0) = 1 - i\int_{t_0}^t dt_1 V_I(t_1) U(t_1, t_0) = \sum_{n=0}^\infty \frac{(-i)^n}{n!}\int_{t_0}^t dt_1 \cdots \int_{t_0}^t dt_n T[V_I(t_1)\cdots V_I(t_n)]
    \label{eq:U_expansion}
\end{equation}
T means the time-ordering:
\begin{equation}
    T(V_I(t_1) V_I(t_2)) \equiv 
    \begin{cases}
	V_I(t_1)V_I(t_2)    & t_1 \le t_2   \\
	V_I(t_2)V_I(t_1)    & t_2 \le t_1   \\
    \end{cases}
\end{equation}

With Eq.~\ref{eq:U_expansion}, we have an iterative expression:
\begin{equation}
    \begin{aligned}
	\bra{f} U(t, t_0) \ket{i} 
	&= \bra{f}\ket{i} - i \bra{f} \int_{t_0}^t dt_1 V_I(t_1) U(t_1, t_0) \ket{i}	\\
	&= \delta_{if} - i\sum_m \int_{t_0}^t dt_1 \bra{f}\exp(iH_0 t_1) V \exp(-iH_0 t_1)(t_1)\ket{m} \bra{m} U(t_1, t_0) \ket{i} \\
	&= \delta_{if} - i\sum_m \bra{f}V\ket{m} \int_{t_0}^t dt_1 \exp(i(E_f - E_m)t_1) \bra{m}U(t_1, t_0)\ket{i}  \\
    \end{aligned}
    \label{eq:finite_S-matrix}
\end{equation}

Truncate Eq.~\ref{eq:finite_S-matrix} into the first order ($\bra{m} U(t_1, t_0) \ket{i} = \delta_{im}$)
and define $T_{if} = \bra{f} V \ket{i}$, we write:
\begin{equation}
    \bra{f} U(t, t_0) \ket{i} = \delta_{if} - iT_{if} \int_{t_0}^t dt_1 \exp(i(E_f - E_i)t_1)
\end{equation}
and 
\begin{equation}
    \begin{aligned}
	S_{if} &= \lim_{t\rightarrow +\infty}\lim_{t_0 \rightarrow -\infty} \bra{f} U(t, t_0) \ket{i}    \\
	    &= \delta_{if} - iT_{if} \int_{-\infty}^{\infty} dt_1 \exp(i(E_f - E_i)t_1)	\\
	    &= \delta_{if} + i2\pi\delta(E_f - E_i)T_{if}   \\
    \end{aligned}
\end{equation}
In the matrix form:
\begin{equation}
    S = 1 + i2\pi T
\end{equation}
S being unitary implies
\begin{equation}
    S^\dag S = (1 - i2\pi T^\dag) (1 + i2\pi T) = 1 + i2\pi(T - T^\dag) + (2\pi)^2 T^\dag T = 1
\end{equation}
which reads
\begin{equation}
    T - T^\dag = i (2\pi)T^\dag T = i(2\pi) T T^\dag
\end{equation}
In terms of the matrix element:
\begin{equation}
    \begin{gathered}
    \delta(E_f - E_i)(T_{if} - T^\dag_{if}) = \sum_m i2\pi\delta(E_f - E_m)\delta(E_m - E_i)T_{fm}T^\dag_{mi}	\\
    T_{if} - T^\dag_{if} = \sum_m i2\pi\delta(E_m - E_i)T_{fm}T^\dag_{mi} = ia_{if} \\
    \end{gathered}
\end{equation}
where 
\begin{equation}
    a_{if} = \sum_m (2\pi) \delta(E_m - E_i)T_{fm}T^\dag_{mi}
\end{equation}
is the absorptive part of the transition amplitude $T_{if}$. $\ket{m}$ extends
to all on-shell intermediate states.

The two components of the S-matrix are straightforward to understand.
The constant piece denotes the evolution of a free particle transforming into 
another free particle without any interactions. Naturally, it can only evolve into itself.
The T matrix characterizes the interaction (transition amplitude) between the initial free
particle state $\ket{i}$ and the final free particle state $\ket{f}$. It quantifies
the likelihood of the interaction.

A free particle state can be fully determined by its momentum vector $\vec{p}$
(disregarding spin for now). For an incoming electron $\ket{\vec{p}_i}$, the probability
of it transitioning into the final state of $\ket{\vec{p}_f}$ is given by:
\begin{equation}
    dP = (phase\ space) \times (transition\ probability) = \frac{d\vec{p}_f}{(2\pi)^3} \times |S_{\vec{p}_i\vec{p}_f}|^2
\end{equation}
For a non trivial case of $\ket{f} \ne \ket{i}$, we have:
\begin{equation}
    S_{if} = i2\pi \delta(E_f - E_i)T_{if}
\end{equation}
The differential cross section will be:
\begin{equation}
    d\sigma = \frac{dP}{\CL \Delta t}
\end{equation}
where $\Delta t$ is the interaction time and $\CL$ is the luminosity, indicating 
number of particles hitting the target per unit area per unit time. 
In our case of an incoming plane wave, $\CL = \rho v = v$.
\begin{equation}
    d\sigma = \frac{1}{v\Delta t} \frac{d\vec{p}_f}{(2\pi)^3} 2\pi\delta(E_f - E_i) \left. 2\pi\delta(E_f - E_i)\right|_{E_f = E_i} |T_{if}|^2
\end{equation}
Transform one $\delta$ expression back to the integrating form: 
\begin{equation}
    \left. 2\pi\delta(E_f - E_i) \right|_{E_f = E_i} 
    = \int_{-\infty}^{+\infty} dt \left.\exp(-i(E_f - E_i)t)\right|_{E_f = E_i}
    = \int_{-\infty}^{+\infty} dt 
\end{equation}
Physically, we do not go back or into infinity in time, because the real particle
is a finite wave packet rather than a plane wave. The integration above should 
be finite and close to the interaction time
\begin{equation}
    \int_{-\infty}^{+\infty} dt \rightarrow \Delta t
\end{equation}
Thus we have a defined cross section
\begin{equation}
    d\sigma = \frac{1}{v} \frac{d\vec{p}_f}{(2\pi)^3} 2\pi\delta(E_f - E_i) |T_{if}|^2
\end{equation}
The cross section is proportional to $|T_{if}|^2$, as already known to us.


%%%%%%%%%%%%%%%%%%%%%%%%
\subsubsection{T-Symmetry}
Time symmetry is a fundamental discrete symmetry in physics, which states that physical laws remains
unchanged under the operation of time reversal. Time reversal refers to the reversal
of the time arrow, leading to the progression of time in the opposite direction.
As a consequence of time reversal, various vectors associated with first-order 
time derivatives undergo a sign reversal. This includes quantities like
momentum, angular momentum and magnetic field.

Express the time reversal operation in QM:
\begin{equation}
    \ket{\tilde{\psi}} = \hat{\mathcal{T}} \ket{\psi} 
\end{equation}
where $\hat{\mathcal{T}}: t \rightarrow -t$ is the time reversal operator. 

In terms of the scattering discussed above, a particle will flip its momentum 
and spin (angular momentum) under time reversal, and pick up a phase.
\begin{equation}
    \ket{\tilde{\psi}} = \hat{\mathcal{T}} \ket{\psi_\uparrow(\vec{k})} = \eta\ket{\psi_\downarrow(-\vec{k})}
\end{equation}
$\eta$ is the phase difference. It is expected that two times of time reversal operation 
will transform a state back to itself, which means $|\eta|^2 = 1$. 
The T matrix in terms of the time-reversed states is:
\begin{equation}
    T_{\tilde{i}\tilde{f}} = \bra{\tilde{f}} V \ket{\tilde{i}}
\end{equation}

It is well known that the EM interaction is invariant under time reversal.
\begin{equation}
    |T_{if}|^2 = |T_{\tilde{f}\tilde{i}}|^2 
\end{equation}

With these concepts, one can also define T-odd quantities, which
are proportional to the difference between the magnitude of a regular T element and 
its time-reversed counterpart:
\begin{equation}
    \begin{aligned}
	\text{T-odd} &\propto |T_{if}|^2 - |T_{\tilde{i}\tilde{f}}|^2	\\
	    &= |T_{if}|^2 - |T_{fi}|^2	\\
	    &= |T_{if}|^2 - |T^\dag_{if}|^2	\\
	    &= |T_{if}|^2 - |T_{if} - ia_{if}|^2	\\
	    &= -i(T_{if}a^*_{if} - T^*_{if}a_{if}) - |a_{if}|^2	\\
	    &= 2\text{Im}(T_{if}a^*_{if}) - |a_{if}|^2
    \end{aligned}
    \label{eq:T-odd}
\end{equation}

%%%%%%%%%%%%%%%%%%%%%%%%
\subsubsection{Transverse Asymmetry}
Denote the incoming and outgoing transversely polarized electron as 
$\ket{\vec{k}}$ and $\ket{\vec{k}'}$, the scattering is shown in Fig.~\ref{fig:transverse_scattering}.
\begin{figure}[h!]
    \centering
    \begin{subfigure}[c]{0.4\linewidth}
	\begin{tikzpicture}[scale=0.8]
	    \begin{feynman}[transform shape]
		\vertex (i1) {$e^-$};
		\vertex [right=1.0cm of i1, inner sep=0pt] (spin) {$\odot$};
		\vertex [right=2.3cm of spin] (ip);
		\vertex [right=2.8cm of ip] (i2) {A};
		\vertex [above right = 2cm and 2cm of ip] (o1) {$e^-$};
		\vertex [below left = 2cm and 2cm of ip] (o2) {A};

		\diagram* { {[edge=fermion]
		    (spin) --[edge label=$\vec{k}$] (ip) [dot] --[edge label = $\vec{k}'$] (o1),
		    (i2) --[edge label=$\vec{p}$](ip) [dot] -- [edge label = $\vec{p}'$]  (o2)},
		    (i1) -- (spin)
		};
	    \end{feynman}
	\end{tikzpicture}
    \end{subfigure}
    \hspace{0.2 cm}
    \textbf{-}
    \hspace{0.5 cm}
    \begin{subfigure}[c]{0.4\linewidth}
	\begin{tikzpicture}[scale=0.8]
	    \begin{feynman}[transform shape]
		\vertex (i1) {$e^-$};
		\vertex [right=1.0cm of i1, inner sep=0pt] (spin) {$\otimes$};
		\vertex [right=2.3cm of spin] (ip);
		\vertex [right=2.8cm of ip] (i2) {A};
		\vertex [above right = 2cm and 2cm of ip] (o1) {$e^-$};
		\vertex [below left = 2cm and 2cm of ip] (o2) {A};

		\diagram* { {[edge=fermion]
		    (spin) --[edge label=$\vec{k}$] (ip) [dot] --[edge label = $\vec{k}'$] (o1),
		    (i2) --[edge label=$\vec{p}$](ip) [dot] -- [edge label = $\vec{p}'$]  (o2)},
		    (i1) -- (spin)
		};
	    \end{feynman}
	\end{tikzpicture}
    \end{subfigure}
    \caption[Feynman diagrams of AT]
    {Feynman diagrams of a transversely polarized electron scatters off
    an unpolarized nuclear target in the COM frame. The vector in or out of the plane
    indicates the electron's spin direction.} 
    \label{fig:transverse_scattering}
\end{figure}

The transverse asymmetry will be:
\begin{equation}
    \CA_n \equiv \frac{N_{\uparrow} - N_{\downarrow}}{N_{\uparrow} + N_{\downarrow}} 
    = \frac{|T_{\uparrow}(\vec{k}, \vec{k}')|^2 - |T_{\downarrow}(\vec{k}, \vec{k}')|^2}{|T_{\uparrow}(\vec{k}, \vec{k}')|^2 + |T_{\downarrow}(\vec{k}, \vec{k}')|^2}
\end{equation}
where $T(\vec{k}, \vec{k}') = \bra{\vec{k}'} V \ket{\vec{k}}$ is the scattering
amplitude and the arrow subscript indicates electron's spin direction.
$T_\downarrow(\vec{k}, \vec{k}')$ is related to $T_\downarrow(-\vec{k}, -\vec{k}')$
by a rotation around the normal direction of the scattering plane, as shown in
Fig.~\ref{fig:rotation_plot}
\begin{figure}[h!]
    \centering
    \begin{subfigure}[c]{0.4\linewidth}
	\begin{tikzpicture}[scale=0.8]
	    \begin{feynman}[transform shape]
		\vertex (i1) {$e^-$};
		\vertex [right=1.0cm of i1, inner sep=0pt] (spin) {$\odot$};
		\vertex [right=2.3cm of spin] (ip);
		\vertex [right=2.8cm of ip] (i2) {A};
		\vertex [above right = 2cm and 2cm of ip] (o1) {$e^-$};
		\vertex [below left = 2cm and 2cm of ip] (o2) {A};

		\diagram* { {[edge=fermion]
		    (spin) --[edge label=$\vec{k}$] (ip) [dot] --[edge label = $\vec{k}'$] (o1),
		    (i2) --[edge label=$\vec{p}$](ip) [dot] -- [edge label = $\vec{p}'$]  (o2)},
		    (i1) -- (spin)
		};
	    \end{feynman}
	\end{tikzpicture}
    \end{subfigure}
    \hspace{0.2 cm}
    $\Longrightarrow$
    \hspace{0.5 cm}
    \begin{subfigure}[c]{0.4\linewidth}
	\begin{tikzpicture}[scale=0.8]
	    \begin{feynman}[transform shape]
		\vertex (i1) {$e^-$};
		\vertex [left=1.0cm of i1, inner sep=0pt] (spin) {$\odot$};
		\vertex [left=2.3cm of spin] (ip);
		\vertex [left=2.8cm of ip] (i2) {A};
		\vertex [below left = 2cm and 2cm of ip] (o1) {$e^-$};
		\vertex [above right = 2cm and 2cm of ip] (o2) {A};

		\diagram* { {[edge=fermion]
		    (spin) --[edge label=$\vec{k}$] (ip) [dot] --[edge label = $\vec{k}'$] (o1),
		    (i2) --[edge label=$\vec{p}$] (ip) [dot] -- [edge label = $\vec{p}'$] (o2)},
		    (i1) -- (spin)
		};
	    \end{feynman}
	\end{tikzpicture}
    \end{subfigure}
    \caption{Rotation by $\pi$ around the normal direction of the scattering plane.} 
    \label{fig:rotation_plot}
\end{figure}
\begin{equation}
    T_\downarrow(\vec{k}, \vec{k}') = e^{i\pi} T_\downarrow(-\vec{k}, -\vec{k}')
\end{equation}

Let $T_{if} = T_{\uparrow}(\vec{k}, \vec{k}')$, then $T_{\tilde{i}\tilde{f}} = T_\downarrow (-\vec{k}, -\vec{k}')$
and
\begin{equation}
    \begin{aligned}
	\CA_n &\approx \frac{|T_{\uparrow}(\vec{k}, \vec{k}')|^2 - |T_{\downarrow}(-\vec{k}, -\vec{k}')|^2}{2|T_{\uparrow}(\vec{k}, \vec{k}')|^2} \\
	    &= \frac{|T_{if}|^2 - |T_{\tilde{i}\tilde{f}}|^2}{2|T_{if}|^2}  \\
	    &= \frac{2\text{Im}(T_{if}a^*_{if}) - |a_{if}|^2}{2|T_{if}|^2}
    \end{aligned}
    \label{eq:transverse_asymmetry}
\end{equation}

We find that the transverse asymmetry is a T-odd quantity. For the EM interaction
\begin{equation}
    T_{if} \propto \alpha \qquad a_{if} \propto \alpha^2
\end{equation}
Because $\alpha \simeq \frac{1}{137}$ is small, we can expand Eq.~\ref{eq:transverse_asymmetry} 
in order of $\alpha$. To the lowest order
\begin{equation}
    \CA_n = 0
    \label{eq:AT_0}
\end{equation}
and to the first order 
\begin{equation}
    \CA_n = \frac{\text{Im}(T_{if}a^*_{if})}{|T_{if}|^2}
    \label{eq:AT_1}
\end{equation}

$T_{ij}$ corresponds the OPE interaction, while $a_{ij}$ represents
the TPE interaction. Therefore, the physical interpretation of
Eq.~\ref{eq:AT_0} and \ref{eq:AT_1} is as follows: the time symmetry requires 
the transverse asymmetry to be zero under the Born approximation (OPE only)
and the (lowest order) non-zero transverse asymmetry comes from the interference 
between OPE and TPE.
\begin{figure}[!h]
    \centering
    \includegraphics[width=0.8\linewidth]{at/OPE_n_TPE}
    \caption{Feynman diagrams of the OPE (left) and TPE (right) interactions.}
\end{figure}

%%%%%%%%%%%%%%%%%%%%%%%%%%%%%%%%%%%%%%%%%%%%%%%%%%%%%%%%%%%%%%%%%%%%%%%%
\section{Measurement of the Transverse Asymmetry: the Method}
The experimentally measured transverse asymmetry is given by:
\begin{equation}
    \CA_{\text{mea}} = \CA_n \vec{\CP}_e \cdot \hat{n} = \CA_n \CP_n \sin(\phi_s - \phi_e) = \CA_n \CP_n \sin\phi
    \label{eq:measured_AT}
\end{equation}
where $\vec{\CP}_e$ is the electron spin vector, whose magnitude is the polarization;
$\CP_n$ denotes the transverse component of the electron spin;
$\phi_s$ being the angle between the electron spin vector and the lab horizontal plane;
$\hat{n} = \frac{\vec{k} \times \vec{k}'}{|\vec{k} \times \vec{k}'|}$ 
is the unit normal vector of the scattering plane;
$\phi_e$ represents the angle between the scattering plane and the horizontal plane;
and $\phi = \phi_s - \phi_e$ refers to the angle between the electron spin vector 
and the scattering plane.
These quantities are depicted in Fig.~\ref{fig:AT_scattering}.
\begin{figure}[h!]
    \centering
    \includegraphics[width=0.7\linewidth]{AT_scattering}
    \caption{Schematic plot of the scattering of a transversely polarized electron.}
    \label{fig:AT_scattering}
\end{figure}

Eq.~\ref{eq:measured_AT} shows the angle dependence of the transverse asymmetry. 
Experimentally, it is convenient to choose the angle $\phi$ to be $90^\circ$. 
By selecting the lab horizontal plane as the scattering plane, the electron spin will
be vertical, being perpendicular to the scattering plane, as adopted in PREX-II and CREX. 
Detailed dynamics for the AT measurement are listed out in Table~\ref{tab:AT_dynamics}.

\begin{table}
    \centering
    \begin{tabular}{c c | c c c}
	\hline
	Exp (Energy)	& Target    & $\langle \theta \rangle ({}^\circ)$   & $\langle Q^2 \rangle \ (\mathrm{GeV}^2)$	& $\langle \sin\phi \rangle$	\\
	\hline
	\multirow{3}{*}{PREX-II (0.95~GeV)}
	    & \Carbon    & 4.87  & 0.0067    & 0.967 \\ 
	    & \ca   & 4.81  & 0.0067    & 0.964 \\ 
	    & \Pb   & 4.69  & 0.0064    & 0.966 \\ 
	\hline
	\multirow{4}{*}{CREX (2.18~GeV)}
	    & \Carbon    & 4.77  & 0.033	& 0.969 \\ 
	    & \ca   & 4.55  & 0.031	& 0.970 \\ 
	    & \Ca   & 4.53  & 0.031     & 0.970 \\ 
	    & \Pb   & 4.60  & 0.032     & 0.969 \\ 
	\hline
    \end{tabular}
    \caption{Dynamics of the AT measurement in PREX-II/CREX.}
    \label{tab:AT_dynamics}
\end{table}

To achieve transverse polarization, a modified configuration of the double 
Wien filters is needed. In this setup, the focus is on rotating the spin to the vertical direction using the vertical Wien filter. The subsequent rotations typically performed for longitudinal polarization, as depicted in Fig.~\ref{fig:double_wien_filters}, are omitted in this case. Specifically, the rotating angle of the spin solenoid is set to approximately 0 degrees.
Since the spin is parallel or antiparallel to the magnetic field within the accelerator arc region, there is no spin precession as observed in the case of longitudinal polarization. This configuration allows for the desired transverse polarization of the electron beam.

In terms of the measurement of transverse polarization, neither Moller, nor Compton
polarimeter was used, because their analyzing powers go to zero in the limit of transverse
polarization. Without a direct measurement of the polarization in the hall, 
we turned to the Mott polarimeter in the injector. 
As said above, the beam transportation from the injector to Hall A
is symmetric and flat, which means the vertical component of the polarization
is preserved (can be safely assumed $>99.9\%$), so the measurement in the
injector can be used as a reliable proxy for the polarization measurement in the hall.

Apart from the configuration variance in the double Wien filters, everything 
else remains the same as in the case of longitudinal polarization with the Compton Chicane
turned off. 

%%%%%%%%%%%%%%%%%%%%%%%%
\subsubsection{Polarization Measurement}
The 5~MeV Mott polarimeter was used to verify the transverse polarization,
which gave about 87\% transverse polarization for both PREX-II and CREX runs.
The Mott data is summarized in Table~\ref{tab:AT_Mott}.
% https://logbooks.jlab.org/entry/3781205, https://logbooks.jlab.org/entry/3718518, 
% spin rotator setting: https://logbooks.jlab.org/entry/4070199
\begin{table}[!h]
    \begin{tabular}{c c c | c c | c}
	\hline
	exp & run & IHWP  & UD (\%)	& LR (\%)   & Vertical Pol (\%)	\\
	\hline
	\multirow{2}{*}{PREX-II}
	    & 8966  & OUT   & $0.0704732 \pm 0.435101$	& $-34.193 \pm 0.418556$	& -87.2048  \\
	    & 8967  & IN    & $0.421465 \pm 0.432328$	& $33.9616 \pm 0.419132$	&  86.6146  \\
	\hline
	\multirow{5}{*}{CREX}    
	    & 9081  & IN    & $-1.16128 \pm 0.334165$   & $-34.1276 \pm 0.325363$	& -87.0380  \\
	    & 9082  & OUT   & $-0.105704 \pm 0.328932$	& $34.0755 \pm 0.324116	$&  86.9051  \\
	    & 9083  & IN    & $-0.613295 \pm 0.333657$	& $-34.3502 \pm 0.32453	$& -87.6057  \\
	    & 9084  & OUT   & $-0.0248337 \pm 0.326988$	& $34.4674 \pm 0.318313	$&  87.9046  \\
	    & 9085  & IN    & $-1.15795 \pm 0.33341 $   & $-34.0401 \pm 0.32742	$& -86.8148  \\
	\hline
    \end{tabular}
    \caption[Mott measurement during AT]
    {Mott measurements during PREX-II and CREX AT runnings. The Up-Down
    asymmetry measures the horizontal transverse polarization while the Left-Right
    asymmetry measures the vertical transverse polarization; The Mott analyzing 
    power is $\CA_{\text{Mott}} = 0.3921 \pm 0.0016$, so the vertical polarization is 
    $\frac{\CA_{\text{LR}}}{\CA_{\text{Mott}}}$.} 
    \label{tab:AT_Mott}
\end{table}

The actual value we used for the transverse asymmetry calculation was the 
average longitudinal polarization measured shortly before and after the AT runs, 
with confidence in our Wien filter settings and that the accelerator 
would not change the beam transverse polarization. 
The polarization results are shown in Table~\ref{tab:AT_polarization}
\begin{table}[!h]
    \centering
    \begin{tabular}{c | c c c}
    \hline
    Exp	& Compton (\%)	& Moller (\%)	& $P_n$ (\%) \\
    \hline
    PREX-II & $88.5533 \pm 0.447$   & $89.67 \pm 0.8$	& $89.7 \pm 0.8$  \\
    CREX    & $86.67 \pm 0.63$	& $86.897 \pm 0.778$	& $86.8 \pm 0.6$  \\
    \hline
    \end{tabular}
    \caption[AT Polarization]
    {The Compton and Moller polarization measurements near the AT runs. 
    The PREX-II AT uses only the Moller result while the CREX one uses the average value of the 
    Compton and the Moller measurements.}
    \label{tab:AT_polarization}
% FIXME why only the moller result for PREX-II
\end{table}

%%%%%%%%%%%%%%%%%%%%%%%%%%%%%%%%%%%%%%%%%%%%%%%%%%%%%%%%%%%%%%%%%%%%%%%%
\section{Data}

% how much data is needed?
We spent one (two) day in PREX-II (CREX) for the transverse asymmetry measurement,
and collected 25 (56) good AT runs in PREX-II (CREX).
% PREX-II: 20190813 - 20190814
% CREX: 20200210 - 20200212

\begin{table}[!h]
    \centering
    \begin{tabular}{c | c | c | c | l}
	\hline
	exp & target	& IHWP	& \# runs    & run number    \\
	\hline
	\multirow{6}{*}{PREX-II}    & \multirow{2}{*}{\Carbon}   & IN    & 3	& 4106-4107, 4133    \\
	    &   & OUT   & 4 & 4108-4109, 4131-4132   \\
	    \cline{2-5}
	    & \multirow{2}{*}{\Pb}  & IN    & 7	& 4115-4119, 4129-4130  \\
	    &	& OUT	& 6 & 4110-4114, 4128   \\
	    \cline{2-5}
	    & \multirow{2}{*}{\ca}  & IN    & 3	& 4123-4125	\\
	    &	& OUT	& 2 & 4126-4127 \\
	\hline
	\multirow{8}{*}{CREX}	& \multirow{2}{*}{\Ca}	& IN	& 9 & 6344-6345,6354-6355,6380-6382,6407-6408\\
	    &	& OUT	& 10	& 6346-6348,6356-6357,6383-6385,6405-6406   \\
	    \cline{2-5}
	    & \multirow{2}{*}{\ca}	& IN	& 7 & 6351-6352,6394-6396,6401-6402	\\
	    &	& OUT	& 7 & 6349-6350,6398-6400,6403-6404	\\
	    \cline{2-5}
	    & \multirow{2}{*}{\Carbon}	& IN	& 6 & 6361-6363,6389-6391	\\
	    &	& OUT	& 5 & 6359-6360,6386-6388	\\
	    \cline{2-5}
	    & \multirow{2}{*}{\Pb}	& IN	& 7 & 6367-6371,6377-6378	\\
	    &	& OUT	& 5 & 6372-6376 \\
	\hline
    \end{tabular}
    \caption{AT runs in PREX-II/CREX}
\end{table}

%%%%%%%%%%%%%%%%%%%%%%%%%%%%%%%%%%%%%%%%%%%%%%%%
\subsection{Data Analysis}
Using the data set after 2 respins and following the standard analysis procedure, 
the transverse asymmetry is extracted. As shown in Eq.~\ref{eq:measured_AT},
$\hat{n}$ of the scattering plane in the LHRS and RHRS are opposite to each other.
Consequently, the measured transverse asymmetries exhibit opposite signs in the LHRS 
and RHRS. To combine the measurements from both arms, the asymmetry difference is
utilized instead of the asymmetry average used in the main analysis. 
The asymmetry difference is defined as (up to a `-' sign):
\begin{equation}
    \CA_{\text{dd}} = \frac{\CA_L - \CA_R}{2}
\end{equation}

% no beammod data
A simple cut of \verb|ErrorFlag == 0| is applied to select good quadruplets.
In PREX-II, run 4112 is a long run with its data split into two root files, the
second one contains only a small size of samples and is therefore ignored in our AT analysis.
Additionally, the first minirun of run 4117 is excluded due to the large charge asymmetry
observed in that minirun. 

The selected good quadruplets are initially grouped into miniruns, and
the average value of the asymmetry difference over these miniruns for each target 
is calculated, which is the desired quantity.
Alternatively, the transverse asymmetry can be extracted the histogram filled with
all quadruplets, known as the mulplot. The mean value of the mulplot serve as
the final result. 

Statistically, there is no difference between these two methods.
They use the same data set and weight each sample equivalently, so they can
be used to cross-check each other. As an example, The mulplots and minirun average plots for
CREX \Ca are shown below in Fig.~\ref{fig:AT_crex_Ca48_mulplot} and 
Fig.~\ref{fig:AT_crex_Ca48_miniruns}.

\begin{figure}[!h]
    \centering
    \includegraphics[width=\linewidth]{at/mulplot_Ca48}
    \caption[AT \Ca Mulplots]
    {Mulplots for CREX \Ca. The red line is a Gaussian fit. One can clearly 
    see how the false asymmetry correction reduces the width of the distribution (note that
    the first plot has a larger X-range than the other two).}
    \label{fig:AT_crex_Ca48_mulplot}
\end{figure}

\begin{figure}[H]
    \centering
    \includegraphics[width=0.9\linewidth]{at/mini_Ca48_asym_us_dd.mean.png}
    \includegraphics[width=0.9\linewidth]{at/mini_Ca48_reg_asym_us_dd.mean.png}
    \includegraphics[width=0.9\linewidth]{at/mini_Ca48_dit_asym_us_dd.mean.png}
    \caption[AT \Ca asymmetry plot]
    {Sign-corrected minirun-wise scatter plot of the raw, regression-corrected and 
    dithering-corrected transverse asymmetry of \Ca in CREX. Different colors represent
    different IHWP states (in/out). In each plot, the top pad shows the 
    mean and error of the title variable for each minirun, the three fit lines indicate
    the zero order polynomial fit to IHWP=in, IHWP=out and all data points respectively;
    the bottom pad is the pull histogram, which is the ratio of the deviation 
    of each data point from the mean value of all data points divided by the corresponding
    data point's error.
    % Regress with the following 5 BPMs: bpm1X, bpm4aY, bpm4eX, bpm4eY, bpm12X.
    % PREX-regression bpms: bpm4aX/Y, bpm4eX/Y, bpmE
    }
    \label{fig:AT_crex_Ca48_miniruns}
\end{figure}


The minirun mean values and mulplot mean values for each target are summarized in the following
tables.
\begin{table}[!h]
    \scriptsize
    \begin{tabular}{c | r@{ $\pm$ }l r@{ $\pm$ }l r@{ $\pm$ }l | r@{ $\pm$ }l r@{ $\pm$ }l r@{ $\pm$ }l}
	\hline
	\multirow{2}{*}{Target}	& \multicolumn{6}{c|}{Minirun Average (ppb)} & \multicolumn{6}{c}{Mulplot (ppb)}	\\
	\cline{2-13}
	    & \multicolumn{2}{c}{raw}   & \multicolumn{2}{c}{reg}	& \multicolumn{2}{c|}{dit}   & \multicolumn{2}{c}{raw}	& \multicolumn{2}{c}{reg}   & \multicolumn{2}{c}{dit}	\\
	\hline
	\multicolumn{13}{c}{IHWP IN}   \\
	\hline
	\Carbon	& 4205.3	& 1113.7    & 5173.9	& 501.5	    & 5169.4	& 502.2	    & 4129.8  & 1117.7    & 5105.0  & 504.9     & 5103.5  & 505.6   \\ 
	\ca	& 3055.8	& 1762.8    & 5507.5	& 419.3     & 5517.4	& 420.7	    & 2979.8  & 1763.3    & 5501.8  & 420.0     & 5511.7  & 421.4   \\   
	\Pb	& -266.2	& 974.3     & 54.9  	& 183.5     & 27.6  	& 185.7	    & -381.0  & 975.2     & 56.0   & 183.5	& 22.4    & 185.8   \\   
	\hline
	\multicolumn{13}{c}{IHWP OUT}   \\
	\hline
	\Carbon	& 6111.0	& 992.8	    & 5685.5	& 437.4	    & 5740.6	& 437.9	    & 6055.9  & 995.8     & 5619.1  & 440.1     & 5669.4  & 440.6   \\ 
	\ca	& 5707.3	& 1687.7    & 5069.0	& 397.0     & 5093.7	& 399.7	    & 5775.6  & 1687.8    & 5033.6  & 396.7     & 5033.6  & 399.2   \\   
	\Pb	& -132.7	& 927.4     & -52.3 	& 176.1     & -26.1 	& 178.7	    & \multicolumn{6}{c}{\color{red}{data cannot be reproduced due to lost of run 4112}}	\\   
	\hline
	\multicolumn{13}{c}{COMBINED}   \\
	\hline
	\Carbon	& 5267.8	& 740.8	    & 5464.5	& 329.6	    & 5493.9	& 330.0	    & 5209.7  & 743.5     & 5393.2  & 331.8     & 5427.0  & 332.2   \\ 
	\ca	& 4439.2	& 1219.1    & 5276.3	& 288.3     & 5294.7	& 289.8	    & 4444.9  & 1219.7    & 5275.3  & 288.5     & 5293.9  & 290.0   \\   
	\Pb	& -196.2	& 671.8     & -0.9  	& 127.0     & -0.3  	& 128.8	    & \multicolumn{6}{c}{\color{red}{data cannot be reproduced due to lost of run 4112}}  \\   
	\hline
    \end{tabular}
    \caption{PREX-II raw and corrected transverse asymmetry}
\end{table}

\begin{table}[!h]
    \scriptsize
    \begin{tabular}{c | r@{ $\pm$ }l r@{ $\pm$ }l r@{ $\pm$ }l | r@{ $\pm$ }l r@{ $\pm$ }l r@{ $\pm$ }l}
	\hline
	\multirow{2}{*}{Target}	& \multicolumn{6}{c|}{Minirun Average (ppb)} & \multicolumn{6}{c}{Mulplot (ppb)}	\\
	\cline{2-13}
	    & \multicolumn{2}{c}{raw}   & \multicolumn{2}{c}{reg}	& \multicolumn{2}{c|}{dit}   & \multicolumn{2}{c}{raw}	& \multicolumn{2}{c}{reg}   & \multicolumn{2}{c}{dit}	\\
	\hline
	\multicolumn{13}{c}{IHWP IN}   \\
	\hline
	\Carbon	& 6815.5    & 1397.2	& 7767.7    & 1182.2	& 7660.5    & 1183.4	& 6885.1    & 1397.9	& 7725.7    & 1182.1	& 7618.8 & 1183.3	\\ 
	\ca     & 8661.9    & 1643.5	& 8777.5    & 1265.2	& 8764.4    & 1267.5	& 8581.7    & 1645.3	& 8743.9    & 1265.3	& 8733.3 & 1267.6	\\ 
	\Ca     & 8306.5    & 1523.3   	& 7677.5    & 1188.9	& 7575.2    & 1190.3	& 8275.7    & 1524.9	& 7658.9    & 1189.0	& 7553.5 & 1190.3	\\ 
	\Pb	& 2742.6    & 2469.1   	& 3052.4    & 2285.9	& 3079.7    & 2288.1	& 2771.1    & 2469.6	& 3101.8    & 2286.2	& 3129.9 & 2288.3	\\ 
	\hline
	\multicolumn{13}{c}{IHWP OUT}   \\
	\hline
	\Carbon	& 8607.9	& 1558.2    & 8789.1	& 1313.5    & 8791.5	& 1314.6    & 8512.9    & 1558.8	& 8778.2    & 1313.6	& 8780.0    & 1314.7	\\      
	\ca      & 8023.6	& 1751.5    & 7967.4	& 1353.3    & 7994.2	& 1355.0    & 8168.4    & 1755.1	& 7960.2    & 1353.4	& 7987.0    & 1355.2	\\      
	\Ca      & 7267.1	& 1516.3    & 8257.8	& 1180.2    & 8254.7	& 1183.3    & 7184.5    & 1517.6	& 8267.8    & 1180.3	& 8270.3    & 1183.5	\\      
	\Pb	& 2089.1	& 2456.4    & 2420.2	& 2263.4    & 2456.9	& 2266.2    & 2075.1    & 2456.8	& 2401.2    & 2263.8	& 2440.7    & 2266.6	\\      
	\hline
	\multicolumn{13}{c}{COMBINED}   \\
	\hline
	\Carbon	& 7614.4	& 1040.3    & 8224.8	& 878.7     & 8166.8	& 879.5	    & 7600.8    & 1040.8	& 8235.1    & 878.8 	& 8177.3    & 879.6	\\      
	\Ca      & 8363.1	& 1198.5    & 8399.7	& 924.2     & 8405.0	& 925.6	    & 8377.3    & 1200.4	& 8383.5    & 924.3 	& 8390.4    & 925.7 	\\        
	\Ca  	& 7784.4	& 1074.7    & 7969.8	& 837.6     & 7916.9	& 839.2	    & 7725.4    & 1075.7	& 7974.4    & 837.7 	& 7923.5    & 839.3 	\\        
	\Pb      & 2414.2	& 1741.4    & 2733.1	& 1608.4    & 2765.3	& 1610.1    & 2422.6    & 1741.7	& 2751.0    & 1608.6	& 2784.8    & 1610.4	\\        
	\hline
    \end{tabular}
    \caption{CREX raw and corrected transverse asymmetry}
\end{table}

\begin{comment}
    & 343.4 & 154.8 & 155.1
    & 379.9 & 91.0  & 91.4
    & 493.9 & 93.0  & 94.1
    & 345.4 & 152.9 & 153.0
    & 387.2 & 91.3  & 91.9
    & 495.3 & 93.5  & 95.3
    & 344.5 & 153.7 & 153.9
    & 383.6 & 91.2  & 91.7
    & 494.5 & 93.2  & 94.6


\begin{table}
    \scriptsize
    \begin{tabular}{c | c c c | c c c}
	\hline
	\multirow{2}{*}{Target}	& \multicolumn{3}{c|}{Minirun Average (ppm)} & \multicolumn{3}{c}{Mulplot (ppm)}	\\
	\cline{2-7}
	    & raw	& reg	& dit	& raw	& reg	& dit	\\
	\hline
	\multicolumn{7}{c}{IHWP IN}   \\
	\hline
	C	& 659.82  & 558.10  & 558.71  & 659.84  & 557.97  & 558.56	\\
	Ca40    & 933.96  & 717.96  & 719.28  & 933.69  & 718.04  & 719.32	\\
	Ca48    & 994.35  & 775.30  & 776.19  & 994.78  & 775.61  & 776.50	\\
	Pb	& 1262.78 & 1168.89 & 1170.02 & 1261.95 & 1168.23 & 1169.35	\\
	\hline
	\multicolumn{7}{c}{IHWP OUT}   \\
	\hline
	C	& 8607.92 & 1558.19	& 8789.05 & 1313.51	& 8791.48 & 1314.60	 \\
	Ca40    & 8023.61 & 1751.48	& 7967.37 & 1353.29	& 7994.17 & 1355.00	 \\
	Ca48    & 7267.11 & 1516.31	& 8257.84 & 1180.23	& 8254.72 & 1183.33	 \\
	Pb	& 2089.10 & 2456.43	& 2420.15 & 2263.44	& 2456.87 & 2266.23	 \\
	\hline
	\multicolumn{7}{c}{COMBINED}   \\
	\hline
	C	& 661.92  & 558.72  & 559.27  & 661.73  & 558.75  & 559.29	\\
	Ca40    & 932.99  & 718.38  & 719.52  & 932.93  & 718.36  & 719.47	\\
	Ca48    & 996.46  & 775.93  & 777.46  & 996.67  & 776.15  & 777.63	\\
	Pb	& 1260.29 & 1163.94 & 1165.22 & 1259.54 & 1163.30 & 1164.57	\\
	\hline
    \end{tabular}
    \caption{Mini-wise average and mulplot average values for each target}
\end{table}
\end{comment}
As shown in the above tables, the final results from the two false asymmetry 
correction methods -- regression and dithering, agree with each other.
We chose the dithering corrected values to extract the transverse asymmetry.	% FIXME why?
The slug-wise plots of the transverse asymmetry are shown in Fig.~\ref{fig:AT_slug} 
\begin{figure}[!h]
    \centering
    \includegraphics[width=\linewidth]{at/prex_at_slug}
    \includegraphics[width=\linewidth]{at/crex_at_slug}
    \caption{Sign-corrected transverse asymmetry in chronological order. 
    Each datapoint represents one slug.}
    \label{fig:AT_slug}
\end{figure}

%%%%%%%%%%%%%%%%%%%%%%%%%%%%%%%%%%%%%%%%%%%%%%%%
\subsection{Systematic Uncertainties}
Various corrections applied to the raw asymmetry introduce associated uncertainties. 
Such as the beam false asymmetry correction, purity and detector/monitor non-linearity correction. 
These uncertainties have an impact on the precision of our measurement, and it is
crucial to have accurate knowledge of these uncertainties.


%%%%%%%%%%%%%%%%%%%%%%%%
\subsubsection{Beam Correction}
% https://prex.jlab.org/DocDB/0005/000504/002/Dithering%20and%20Regression.pdf
To quantify the uncertainty arising from the beam correction, we compare 
the corrections obtained using the regression and dithering methods. Specifically, we observe that for the majority of runs, the difference between the corrections derived from the most significant BPMis using these two methods is below 5\%. Consequently, we conservatively estimate the systematic uncertainty of the beam false asymmetry correction as 5\% of the correction obtained with the dithering method.

The correction from each BPM (or their combinations) is calculated as the product of the target-wise dithering slope and the average difference for each BPM (or their combinations). The uncertainty is obtained by taking the RMS of 5\% of these individual corrections. The results are presented in Table~\ref{tab:beam_correction_uncertainty}.

\begin{table}[!h]
    \centering
    \begin{tabular}{c c | c c | c c | c c | c}
	\hline
	Exp & Target	
	& \multicolumn{2}{c|}{\thead{$\CA_{\text{raw}} \pm d\CA_{\text{raw}}$ \\ (ppb)}}    
	& \multicolumn{2}{c|}{\thead{$\CA_{\text{dit}}  \pm d\CA_{\text{dit}}$ \\  (ppb)}}	
	& \multicolumn{2}{c|}{\thead{$\Delta\CA \pm d(\Delta\CA)$ \\ (ppb)}}	
	& $d\Delta\CA/d\CA_{\text{dit}}$\\
	\hline
	\multirow{3}{*}{PREX-II}
	    & \Carbon    & -5268	& 741	& -5494	& 330	& 226.1	& 29.4	& 9\%	\\ 
	    & \ca   & -4439	& 1219	& -5295	& 290	& 195.9 & 42.4	& 15\%	\\ 
	    & \Pb   & 196.2	& 672	& 0.257	& 129	& 855.5 & 71.0	& 55\%	\\ 
	\hline
	\multirow{4}{*}{CREX}
	    & \Carbon    & -7614	& 1040	& -8167	& 880	& 552.4 & 37.8	& 4\%	\\ 
	    & \ca   & -8363	& 1198	& -8405	& 926	& 351.1 & 48.9	& 5.3\%	\\ 
	    & \Ca   & -7784	& 1075	& -7917	& 839	& 41.9  & 86.7	& 10\%	\\ 
	    & \Pb   & -2414	& 1741	& -2765	& 1610	& 132.5 & 27.8	& 2\%	\\ 
	\hline
    \end{tabular}
    \caption{Beam correction to transverse asymmetry.}
    \label{tab:beam_correction_uncertainty}
\end{table}

%%%%%%%%%%%%%%%%%%%%%%%%
\subsubsection{Purity Correction}
For the target purity correction, we need to consider only the \Pb and \Ca targets.
As will be discussed in the following chapter, the contamination in the \Pb target
arises from the diamond foils sandwiching the \Pb foil to cool the target. On
the other hand, the main impurity in \Ca target is the \ca isotope. 
The \ca target has an abundance larger than 99.6\%, so it was regarded as a pure target.

\begin{equation}
    \begin{gathered}
	\CA_{\text{mea}} = \frac{R_t\CA_t + \sum_i R_i \CA_i}{R_t + \sum_i R_i} = \frac{\CA_t + \sum_i f_i \CA_i}{1 + \sum_i f_i}  \\
	\CA_t = (1 + \sum_i f_i)\CA_{\text{mea}} - \sum_i f_i\CA_i
    \end{gathered}
    \label{eq:AT-asymmetry_correction}
\end{equation}
where $R$ and $\CA$ are the scattering rate and the transverse asymmetry for each 
target nucleus. The subscript $t$ and $i$ refer to the target and various 
impurity elements present in the target, respectively. 
$f_i = \frac{R_i}{R_t}$ is the rate fraction. 
We used simulations to calculate the scattering rate for each different
target and the asymmetry values came from measurements directly. The diamond (C) rate fraction
in the \Pb target is:
\begin{equation}
    f_C = \frac{R_C}{R_{Pb}} = 
    \begin{cases}
	0.0671 \pm 0.0057   & E = 0.95\ GeV	\\
	0.6089 \pm 0.0609   & E = 2.2\ GeV	\\
    \end{cases}
\end{equation}

The \Ca case is a little complicated, because the \Ca target is a stack of 3 different
pieces with varying purities. The upstream two pieces are the remnants of the 
old target with a \Ca abundance of 95.99\%, while the downstream piece is a new foil
with a \Ca abundance of 90.04\%. 

Considering that the contamination primarily arises from 
from different isotopes of \Ca, namely \ca ($\sim10\%$), ${}^{42}$Ca ($\sim0.1\%$) 
and ${}^{44}$Ca ($\sim0.2\%$), 
whose scattering rates and asymmetries are similar to that of \Ca, so we simply 
count the non-\Ca fraction in the \Ca target, which leads to:
\begin{equation}
    f(\frac{non-{}^{48}Ca}{{}^{48}Ca}) = 9.07 \pm 0.18 \%
\end{equation}

Using equation \ref{eq:AT-asymmetry_correction}, the asymmetry after the purity correction
is shown in Table~\ref{tab:purity_corrected_asymmetry} .
\begin{table}
% https://docs.google.com/spreadsheets/d/1ZI68PgAn_zySKozZ__kHBvxlBmaTMKw9jPSl-EIa91k/edit?pli=1#gid=1243115322
% why cell J4 does not follow error propagation
% K3-K9: the formula looks weird
    \centering
    \begin{tabular}{c c | c c }
	\hline
	Exp & Target	
	& \multicolumn{2}{c}{$\CA_{\text{cor}}  \pm d\CA_{\text{stat}}$ (ppb)}	    \\
	\hline
	\multirow{3}{*}{PREX-II}
	    & \Carbon    & -5494	& 330	 \\ 
	    & \Ca   & -5295	& 290	 \\ 
	    & \Pb   & 369	& 137	 \\ 
	\hline
	\multirow{4}{*}{CREX}
	    & \Carbon    & -8167	& 880	 \\ 
	    & \ca   & -8405	& 926	 \\ 
	    & \Ca   & -7873	& 919	 \\ 
	    & \Pb   & 523	& 2646	 \\ 
	\hline
    \end{tabular}
    \caption{Purity-corrected transverse asymmetry. The statistical uncertainties
    are calculated following the uncertainty propagation equation.}
    \label{tab:purity_corrected_asymmetry}
\end{table}

%%%%%%%%%%%%%%%%%%%%%%%%
\subsubsection{Detector Non-linearity}
For the uncertainty caused by the non-linearity in the detector's response 
to the incoming electron flux, it is constrained to be $<0.5\%$ based on bench tests.
\begin{table}[!h]
    \centering
    \begin{tabular}{c c | c c c}
	\hline
	Exp & Target	& $\CA_{\text{raw}}$ (ppb) & $d\CA_{\text{sys}}$ (ppb)    & $\frac{d\CA_{\text{sys}}}{\CA_{\text{raw}}}$   \\
	\hline
	\multirow{3}{*}{PREX-II}
	    & \Carbon    & -5268	& 26	& 0.50\%    \\ 
	    & \ca   & -4439	& 22	& 0.50\%    \\ 
	    & \Pb   & 196.2	& 1	& 0.50\%    \\ 
	\hline
	\multirow{4}{*}{CREX}
	    & \Carbon    & -7614	& 38	& 0.50\%    \\ 
	    & \ca   & -8363	& 42	& 0.50\%    \\ 
	    & \Ca   & -7784	& 39	& 0.50\%    \\ 
	    & \Pb   & -2414	& 12	& 0.50\%    \\ 
	\hline
    \end{tabular}
    \caption{Systematic uncertainty due to the detector non-linearity.}
\end{table}

For uncertainty come from the BCM non-linearity, a conservative estimate of 1\% was used,
as shown in Table~\ref{tab:AT_bcm_non-linearity}, the charge asymmetry is the 
minirun-wise average value.
\begin{table}[!h]
    \centering
    \begin{tabular}{c c | c c c}
	\hline
	Exp & Target	& $\CA_{q}$ (ppb) & $d\CA_{q}$ (ppb)    & $\frac{d\CA_{q}}{\CA_{q}}$   \\
	\hline
	\multirow{3}{*}{PREX-II}
	    & \Carbon    & -52.863   & 0.5   & 1.00\%    \\ 
	    & \ca   & -104.763  & 1.0   & 1.00\%    \\ 
	    & \Pb   & 140.602   & 1.4   & 1.00\%    \\ 
	\hline
	\multirow{4}{*}{CREX}
	    & \Carbon    & 50.09	& 0.5   & 1.00\%    \\ 
	    & \ca   & 47.81	& 0.5   & 1.00\%    \\ 
	    & \Ca   & 27.35	& 0.3   & 1.00\%    \\ 
	    & \Pb   & -1.61	& 0.0   & 1.00\%    \\ 
	\hline
    \end{tabular}
    \caption{Systematic uncertainty due to the BCM non-linearity}
    \label{tab:AT_bcm_non-linearity}
\end{table}

%%%%%%%%%%%%%%%%%%%%%%%%%%%%%%%%%%%%%%%%%%%%%%%%
\subsection{Dynamics}

%%%%%%%%%%%%%%%%%%%%%%%%
\subsubsection{$\phi$ Angle}
% http://ace.phys.virginia.edu/HAPPEX/4179
% http://ace.phys.virginia.edu/HAPPEX/4179
It is said above that we aimed to choose the angle $\phi$ to be exactly $90^\circ$.
However, it is practically impossible. The actually value typically deviates slightly from
the designed value, as determined from the data. By drawing the $\sin\phi$ distribution
from data and calculate the average from the distribution, the measured value of $\sin\phi$ 
is determined, which is shown in the following table.
\begin{table}[!htbp]
    \centering
    \begin{tabular}{c c | c c c}
	\hline
	Exp & Target	& LHRS $\sin\phi$   & RHRS $\sin\phi$	& average   \\
	\hline
	\multirow{3}{*}{PREX-II}
	    & \Carbon    & 0.96660   & 0.96700	& 0.9668    \\ 
	    & \ca   & 0.96430   & 0.96440	& 0.9644    \\ 
	    & \Pb   & 0.96625   & 0.96665	& 0.9665    \\ 
	\hline
	\multirow{4}{*}{CREX}
	    & \Carbon    & 0.96950   & 0.96790	& 0.9687    \\ 
	    & \ca   & 0.97090   & 0.96920	& 0.9701    \\ 
	    & \Ca   & 0.97110   & 0.96880	& 0.9700    \\ 
	    & \Pb   & 0.96980   & 0.96830	& 0.9691    \\ 
	\hline
    \end{tabular}
    \caption{Average $\sin\phi$ values for different AT targets.}
\end{table}

%%%%%%%%%%%%%%%%%%%%%%%%
\subsubsection{$Q^2$}
Similar to the extraction of the $\phi$ angle, we drew the $Q^2$ distribution for
each target, and then took the mean value. The results are shown in the
following table.
% http://ace.phys.virginia.edu/HAPPEX/4453
% http://ace.phys.virginia.edu/HAPPEX/4468
\begin{table}[!htbp]
    \centering
    \begin{tabular}{c c | r@{ $\pm$ }l | r@{ $\pm$ }l | r@{ $\pm$ }l r@{ $\pm$ }l}
	\hline
	Exp & Target	
	& \multicolumn{2}{c|}{\thead{LHRS $Q^2$ \\ ($\mathrm{GeV}^2$)}} 
	& \multicolumn{2}{c|}{\thead{RHRS $Q^2$ \\ ($\mathrm{GeV}^2$)}} 
	& \multicolumn{2}{c}{\thead{Average $Q^2$ \\ ($\mathrm{GeV}^2$)}} & \multicolumn{2}{c}{\thead{Average $Q$ \\ ($\mathrm{GeV}$)}} \\
	\hline
	\multirow{4}{*}{PREX-II}
	& \Carbon	& 0.0068    & 4E-6  & 0.0066    & 5E-6	& 0.00671   & 3.21E-6	& 0.082	& 1.96E-5	\\
	& \ca  	& 0.0068    & 5E-6  & 0.0067    & 6E-6  & 0.00673   & 4.17E-6  & 0.082	& 2.54E-5	\\
	& \Pb 8	& 0.0065    & 5E-6  & 0.0063    & 6E-6  & 0.00640   & 4.06E-6  & 0.080	& 2.54E-5	\\
	& \Pb 9	& 0.0065    & 4E-6  & 0.0063    & 5E-6  & 0.00640   & 3.50E-6  & 0.080	& 2.18E-5	\\
	\hline
	\multirow{4}{*}{CREX}
	& \Carbon	& 0.0328    & 2E-5  & 0.0334    & 2E-5	& 0.0331    & 1.31E-5  & 0.182	& 3.61E-5	\\
	& \ca  	& 0.0306    & 2E-5  & 0.0309    & 2E-5	& 0.0308    & 1.22E-5  & 0.175	& 3.48E-5	\\
	& \Ca  	& 0.0304    & 1E-5  & 0.0307    & 2E-5	& 0.0306    & 1.07E-5  & 0.175	& 3.05E-5	\\
	& \Pb	& 0.0319    & 3E-5  & 0.0322    & 3E-5	& 0.0320    & 1.99E-5  & 0.179	& 5.56E-5	\\
	\hline
    \end{tabular}
    \caption{Average $Q^2$ values for different AT targets.}
\end{table}

%%%%%%%%%%%%%%%%%%%%%%%%%%%%%%%%%%%%%%%%%%%%%%%%
\subsection{Final Result}
With Eq.~\ref{eq:measured_AT}, the transverse asymmetry is calculated to be:
\begin{equation}
    \CA_n = \frac{\CA_{\text{cor}}}{\CP_n \cdot \sin\phi}
\end{equation}

The statistical uncertainty is:
\begin{equation}
    d\CA_n(\text{stat}) = \frac{d\CA_{\text{cor}}(\text{stat})}{\CP_n \cdot \sin\phi}
\end{equation}
and the systematic uncertainty is:
\begin{equation}
    \left( \frac{d\CA_n(\text{sys})}{\CA_n} \right)^2 = 
	\left( \frac{d\CA_{\text{cor}}(\text{sys})}{d\CA_{\text{cor}}} \right)^2
	+ \left( \frac{d\CP_n}{\CP_n}\right)^2 
\end{equation}
where
\begin{equation}
    d\CA^2_{\text{cor}}(\text{sys}) = d\CA^2(\text{det\ nonlin}) + d\CA^2(\text{BCM\ nonlin}) + d\CA^2(\text{beam\ correction})
\end{equation}
For the \Pb and \Ca targets, we need to include uncertainties from contaminations.
Various systematic uncertainties are summarized in Table~\ref{tab:AT_uncertainties}.
\begin{table}[!h]
    \centering
    \begin{tabular}{c c c c | c c c c}
	\hline
	Exp & \multicolumn{3}{c|}{PREX-II}  & \multicolumn{4}{c}{CREX}	\\
	Target	& \Carbon	& \ca	& \Pb	& \Carbon	& \ca	& \Ca	& \Pb	\\
	\hline
	Beam correction & 0.03  & 0.05  & 0.08  & 0.04  & 0.06  & 0.10  & 0.03	\\
	Polarization    & 0.06  & 0.05  & $<0.01$ & 0.08  & 0.08  & 0.08  & $<0.01$	\\
	Non-linearity   & 0.03  & 0.03  & $<0.01$ & 0.05  & 0.05  & 0.05  & 0.01	\\
	Tgt. impurity   & $<0.01$ & $<0.01$ & 0.04  & $<0.01$ & $<0.01$ & 0.10  & 0.80	\\
	Inelastic	& $<0.01$ & $<0.01$ & $<0.01$ & 0.08  & 0.15  & 0.08  & $<0.01$	\\
	\hline	
	Tot. Syst	& 0.07  & 0.08  & 0.09  & 0.13  & 0.18  & 0.19  & 0.75	\\
	Statistical	& 0.38  & 0.34  & 0.16  & 1.05  & 1.10  & 1.09  & 3.15	\\
	Total		& 0.39  & 0.34  & 0.18  & 1.05  & 1.11  & 1.11  & 3.23	\\
	\hline
    \end{tabular}
    \caption{AT uncertainty contributions in units of ppm}
    \label{tab:AT_uncertainties}
\end{table}

The final result is shown in Table~\ref{tab:AT_final_values}:
\begin{table}[!h]
    \centering
    \begin{tabular}{c c | c c c c}
	\hline
	Exp & Target	& \thead{$\CA_n$ \\ (ppm)}   & \thead{$d\CA_{stat}$ \\ (ppm)}	
	& \thead{$d\CA_{sys}$ \\ (ppm)}	& \thead{$d\CA_{stat+sys}$ \\ (ppm)}	\\
	\hline
	\multirow{3}{*}{PREX-II}
	    & \Carbon    & -6.34	& 0.38	& 0.07	& 0.39	\\ 
	    & \ca   & -6.12	& 0.34	& 0.08	& 0.34	\\ 
	    & \Pb   & 0.43	& 0.16	& 0.09	& 0.18	\\ 
	\hline
	\multirow{4}{*}{CREX}
	    & \Carbon    & -9.71	& 1.05	& 0.10	& 1.05	\\ 
	    & \ca   & -9.98	& 1.10	& 0.11	& 1.11	\\ 
	    & \Ca   & -9.35	& 1.09	& 0.17	& 1.11	\\ 
	    & \Pb   & 0.62	& 3.15	& 0.75	& 3.23	\\ 
	\hline
    \end{tabular}
    \caption{Final result of the transverse asymmetry.}
    \label{tab:AT_final_values}
\end{table}

Upon comparing the experimental results to theoretical calculations \cite{PhysRevC.103.064316}, 
we confirm the anomaly presented in PREX-I AT measurement. Specifically, 
the transverse asymmetries for \Pb are consistently 0 at various $Q$ values, 
as shown in Fig.~\ref{fig:pcrex_AT}. 
On the other hand, the transverse asymmetries for other light nuclei are relatively
close to their corresponding theoretical predictions.
\begin{figure}[!h]
    \centering
    \includegraphics[scale=.5]{at/pcrex_AT}
    \caption[PCREX AT result]
    {Transverse asymmetries measured in PREX-II/CREX \cite{PhysRevLett.128.142501}. 
    The PREX-I result is also included. Overlapping points are offset slightly in Q to distinguish
    them.}
    \label{fig:pcrex_AT}
\end{figure}

\begin{comment}
    resource:
    \begin{itemize}
	\item C contaminations: https://docs.google.com/spreadsheets/d/1PSR-MtSp1jTthA82ohaNlGB7ZESbbdwzgZsv32p5TvM/edit#gid=0
	\item AT data: https://docs.google.com/spreadsheets/d/1ZI68PgAn_zySKozZ__kHBvxlBmaTMKw9jPSl-EIa91k/edit?pli=1#gid=1243115322
	\item AT plot: https://github.com/cipriangal/prexATplot
    \end{itemize}

    Misc:
    \begin{itemize}
	\item regression: which bpm set were used?
	\item dithering: ???
	\item 
    \end{itemize}
\end{comment}
