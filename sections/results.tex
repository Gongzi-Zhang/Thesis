\chapter{Results}

\section{Polarization}
\subsection{Moller}
\subsection{Compton}

\section{Final Number}
\begin{table}
    \centering
    % \begin{tabular}
    % \end{tabular}
\end{table}

\section{Physical Implication}

%%%%%%%%%%%%%%%%%%%%%%%%%%%%%%%%%%%%%%%%%%%%%%%%
\subsection{Theoretical Models}
Neither the nuclear interaction nor their wave functions are known to us.
Unlike particle physics, there is no such a single Standard Model to describe
general properties of a nuclear system, such as the ground state binding energy,
nuclear size and excitation spectrum. Various models work well in their own
territories. Ab-initio is mainly used to describe light nuclei while Density
Functional Theory (DFT) provides very precise prediction for heavy nuclei.

%%%%%%%%%%%%%%%%%%%%%%%%
\subsubsection{ab-initio}
Ab-initio is more like a theoretical approach, that needs to calculate the force
between nucleons and then solve the quantum many-body equation from them. While
DFT is more like an experimental approach, which uses experimental data to fit
interactions in a given region of the nuclear landscape.
\begin{itemize}
    \item Green's function Monte Carlo (GFMC)
    \item No-Core Shell Model (NCSM)
    \item Coupled-cluster
\end{itemize}
%%%%%%%%%%%%%%%%%%%%%%%%
\subsubsection{Nuclear Density Functional Theory (DFT)}
The basic idea is to construct a general functional, with nucleons density 
distribution as input, will output the ground-state energy and other properties
of the nuclear system, the difficulty lies in that no single general functional
can cover all nuclei.
The basic idea is simple, once we know the density distribution function, then
one can calculate the total energy of the system based on this distribution 
function, minimization of the total energy will be the ground state, and other
static properties will be inferred from the ground state. Excitation properties
can also be calculated from DFT. The only problem is how to know the density
distributio function.


%%%%%%%%%%%%%%%%%%%%%%%%
\subsubsection{Effective Field Theory (EFT)}
Based on nucleons and pions, but still obey the symmetry of QCD. 
% The strength of an EFT approach to thenuclear force lies in its power-counting capability. That is, theLagrangian of a theory with nucleons and pions can be expanded  order  by  order  in  terms  of  momentum  transfer  divided by a parameter that sets the momentum scale of the expansion
Various EFT models are based on an effective interacting Lagragian, for example,
FSUGold model has the following effective Lagrangian \cite{PhysRevLett.95.122501}:
\begin{equation}
    \begin{aligned}
	\CL_{\text{int}} = &\bar{\psi} \left[ g_s\phi - \left( g_v V_\mu + \frac{g_\rho}{2}\vec{\tau}\cdot\vec{b}_\mu + \frac{e}{2}(1 + \tau_3) A_\mu \right)\gamma^\mu \right]\psi \\
	    & - \frac{\kappa}{3!}(g_s\phi)^3 - \frac{\lambda}{4!}(g_s\phi)^4 + \frac{\zeta}{4!}(g_v^2 V_\mu V^\mu )^2	\\
	    & + \Lambda_v(g_\rho^2\vec{b}_\mu\vec{b}^\mu)(g_v^2 V_\mu V^\mu)
    \end{aligned}
\end{equation}
This Lagrangian density descirbes interactions of the nucleon field $\psi$ to
various meson fields and their self-interctions. $\phi$ is a scalar.

The difference between different EFT models is just how many coupling they
include in their effetive Lagrangian density. With the Lagrangian density,
one can calculate the properties of various nuclei, fitting predicted values
to experimental results to get a parameter set for the coupling constant in
the Lagrangian, which is called one model. Frequently used EFT models include
NL3 \cite{}, FSUGold \cite{} and 





%%%%%%%%%%%%%%%%%%%%%%%%
\subsubsection{Saturation}
The invariance of bingding energy per nucleon ($E_b/A$) w.r.t. A means that
the interaction between nucleons is not proportional to $A(A-1)$, but proportional
to A, which means nucleons saturate.
\subsection{Saturation Density}
Nuclear Density Function Theory
the basic idea is given a Lagrangian density function:

%%%%%%%%%%%%%%%%%%%%%%%%
\subsubsection{Atomic Parity Violation Measurement}
Accuracy of atomic PV measurement is about 0.3\% (FIXME), which is important
for the test of the SM and the search for physics beyond the SM. A higher (0.1\%)
precision requires knowledge about the neutron radius better than 1\%. \cite{PhysRevC.46.2587}

%%%%%%%%%%%%%%%%%%%%%%%%%%%%%%%%%%%%%%%%%%%%%%%%%%%%%%%%%%%%%%%%%%%%%%%%
\section{Neutron Stars}
Pb neutron radius is large ==> stiff EOS at low nuclear density (subnuclear density)
combine NS radius measurement
NS radius is small ==> soft EOS at high density
these 2 measurements will mean softening of EOS with density ==> transition to
an exotic high density phase such as quark matter, strange matter, color
superconductor, kaon condensate

%%%%%%%%%%%%%%%%%%%%%%%%
\subsubsection{URCA Cooling}
proton fraction for matter in beat equilibrium depends on symmetry energy S(n).o
The larger Rn in Pb, the lower the threshold mass for direct URCA cooling.
If $R_n - R_p < 0.2 \ fm$ all EOS models don't have direct URCA in 1.4 $M_{sun}$ stars
If $R_n - R_p > 0.25 \ fm$, all models do have URCA in 1.4 $M_{sun}$ stars
