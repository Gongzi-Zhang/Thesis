Despite great leaps of development in nuclear physics over the past century, 
we still don’t have a comprehensive understanding of nuclear structure. This 
is mainly due to the lack of precise knowledge of neutron distribution inside nucleus. 
Electromagnetic probes are ineffective in probing neutrons, unlike their charged partners, 
the protons. Therefore, it is of great importance to constrain the neutron distributions experimentally. 
Heavy nuclei have more neutrons than protons in order to balance the repulsion between protons. 
In such neutron-rich nuclei, the extra neutrons are pushed out to the surface 
by the nuclear symmetry energy, forming the so-called ``neutron skin". 
The neutron skin can be probed with a well-established experimental technique 
-- parity-violating electron scattering (PVES). Using the scattering of 
longitudinally polarized electrons by an unpolarized target, PREX-II and CREX 
measure the small parity-violating asymmetry in cross sections. Employing 
electrons with opposite helicities the weak form factor, the neutron distribution 
and the neutron skin thickness of the target nucleus are extracted. 
With excellent beam qualities and dedicated instrumentation at Jefferson Lab, 
the asymmetry measurements are statistics-limited. We report the results of these 
two high-precision measurements and their implications on broad topics, 
from the nuclear structures to the neutron stars.
