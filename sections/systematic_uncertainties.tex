\chapter{Systematic Uncertainties}

Systematic uncertainty control is very important to and a highlight of this 
high-precision experiment. To achieve a smaller systematic uncertainty, a
combination of fast-control and slow control was employed, which helped us
to eliminate systematic uncertainties brought by the accelerator to the
beam. Except the uncertainty from the machine, another important source
of systematic undertainties come from detection process, namely the 
acceptance function.

%%%%%%%%%%%%%%%%%%%%%%%%%%%%%%%%%%%%%%%%%%%%%%%%%%%%%%%%%%%%%%%%%%%%%%%%
\section{$Q^2$ and $\theta$}
Physical interpretation of $\CA_{PV}$ requires measurement of $Q^2$ better
than 1\%.
Counting mode:
\begin{itemize}
    \item low current
    \item single electron events
    \item tracks 
\end{itemize}

%%%%%%%%%%%%%%%%%%%%%%%%%%%%%%%%%%%%%%%%%%%%%%%%%%%%%%%%%%%%%%%%%%%%%%%%
\section{Carbon Contamination in PREX-II}
The Pb foil is not good thermal conductivity, which limits the highest current
we can apply. To help dissipate the heat produced
by electron bombarment, auxiliary meterials -- the Diamond foils, which are 
excellent thermal conductivity are used. Besides, C12 is isoscalar, and spin-0 
nucleus, whose PV asymmetry is well-measured (FIXME), so the background is 
well-understood. Even so, the highest current is only $70 \ \mu A$.

The Pb208 foil is $0.5\ mm$ thick, each diamond foil is half the thickness 

For CREX, Ca48 has good thermal conductivity, so the current can go up to $150\ \mu A$
the contamination is mainly Ca40 ($\sim 4\%$ FIXME), which is also isoscalar
and spin-0 nucleus, so benign background

\section{Accptance Function}
The acceptance function is defined as the proportion of detected electrons
over scattered electrons, it is a function of the scattering angle $\theta$:
$$ A(\theta) = \frac{N_{det}(\theta)}{N_{sca}} $$

The acceptance function has another importance: only with the acceptance
function, can we interpret our measurement:
\includegraphics[width=0.5\linewidth]{asym_ca48}
\begin{equation*}
    \langle A \rangle = \frac{\int d\theta \sin\theta A(\theta) \frac{d\sigma}{d\Omega} \epsilon(\theta)}{\int d\theta \sin\theta \frac{d\sigma}{d\Omega} \epsilon(\theta)}
\end{equation*}

The idea is to match simulation with optics data; then we can extract the 
acceptance function from the simulation. For optics data, we have 2 sieves
(which is just a thin steel plane with many holes on it) before the septum.
Because we know each hole position on the sieve, we can therefore reconstruct
the transfrom matrices between target and detector. Which can then be used in
simulation to extract the acceptance function.

For the simulation, we need to identify a few things:
\begin{itemize}
    \item Beam Position
    \item 
\end{itemize}

We will scan through some parameters to find the best model:
\begin{itemize}
    \item Septum current
    \item Collimator shift
    \item Pinch point shift
\end{itemize}

% \subsection{tgt variables}
