\chapter{Systematic Uncertainties}

Systematic uncertainty control is very important to and a highlight of this 
high-precision experiment. To achieve a smaller systematic uncertainty, a
combination of fast-control and slow control was employed, which helped us
to eliminate systematic uncertainties brought by the accelerator to the
beam. Except the uncertainty from the machine, another important source
of systematic undertainties come from detection process, namely the 
acceptance function.

%%%%%%%%%%%%%%%%%%%%%%%%%%%%%%%%%%%%%%%%%%%%%%%%%%%%%%%%%%%%%%%%%%%%%%%%
\section{$Q^2$ and $\theta$}
Physical interpretation of $\CA_{PV}$ requires measurement of $Q^2$ better
than 1\%.
Counting mode:
\begin{itemize}
    \item low current
    \item single electron events
    \item tracks 
\end{itemize}

%%%%%%%%%%%%%%%%%%%%%%%%%%%%%%%%%%%%%%%%%%%%%%%%%%%%%%%%%%%%%%%%%%%%%%%%
\section{Carbon Contamination in PREX-II}
The Pb foil is not good thermal conductivity, which limits the highest current
we can apply. To help dissipate the heat produced
by electron bombarment, auxiliary meterials -- the Diamond foils, which are 
excellent thermal conductivity are used. Besides, C12 is isoscalar, and spin-0 
nucleus, whose PV asymmetry is well-measured (FIXME), so the background is 
well-understood. Even so, the highest current is only $70 \ \mu A$.

The Pb208 foil is $0.5\ mm$ thick, each diamond foil is half the thickness 

For CREX, Ca48 has good thermal conductivity, so the current can go up to $150\ \mu A$
the contamination is mainly Ca40 ($\sim 4\%$ FIXME), which is also isoscalar
and spin-0 nucleus, so benign background

%%%%%%%%%%%%%%%%%%%%%%%%%%%%%%%%%%%%%%%%%%%%%%%%%%%%%%%%%%%%%%%%%%%%%%%%
\section{Accptance Function}
As we said before, the spectrometer acceptance was mainly defined by the Q1 
collimator, but other devices may also have affections on it. And the accepted
area is quite large ($0.00377 sr$), not every point within the acceptance has 
the same detection efficiency and cross section asymmetry, therefore what
we measured was actually the average asymmetry over the acceptance:
\begin{equation}
    \CA_{mea} = \frac{\int d\theta \sin\theta A(\theta) \frac{d\sigma}{d\Omega} \epsilon(\theta)}{\int d\theta \sin\theta \frac{d\sigma}{d\Omega} \epsilon(\theta)}
    \ref{eq:acceptance_function}
\end{equation}
Here $\epsilon(\theta)$ is the aceeptance function, which is defined as
the proportion of electrons that reach the main detector over all scattered
electrons, which is a function of the scattering angle $theta$:
$$ \epsilon(\theta) = \frac{N_{det}(\theta)}{N_{sca}} $$

From Eq. \ref{eq:acceptance_function}, we see the importance of the acceptance
function. Firstly, it will be a systematic uncertainty of our asymmetry
measurement; and secondly, only with the acceptance function, can we compare
our experimental measurement to theoretical predictions to interpret our
result.

To extract the acceptance function, we have to turn to simulation. Then how
do we know our simulation was correct? We will compare our simulation result
to optics runs, and match of various kinematic variables between simulation
and data was required. 

When we took optics data, we will put in the sieve slit collimators so that 
we can reconstruct electron trajectory using track info from VDCs or GEMs 
to match holes in the sieve plane, therefore we can know the scattering angle, 
energy for each electron trajectory.

In terms of simulation, we will tune a few parameters to find out the best
match: so called best model, which will be used to calculate the acceptance
function. The few parameters we explored were:
\begin{itemize}
    \item Septum current
    \item Q1 collimator shift
    \item Pinch point shift
\end{itemize}

%%%%%%%%%%%%%%%%%%%%%%%%%%%%%%%%%%%%%%%%%%%%%%%%
\subsection{Data}
Due to the existance of various magnetic fields (septum, HRS) from target to detector, 
there is no way to know the exact analytical expression of the transportation
from target to detectors, we have to measure it.
is array of parameters that define the evolution of charge particle's trajectory 
inside a magnetic environment. The electron's trajectory can be parameterized
as: $(x, \theta, y, \phi, \delta)^T$ w.r.t. to a reference trajectory.
In the transport coordinate, $\hat{z}$ is the direction of reference trajectory;
x is the displacement in the dispersive plane relative to the reference 
trajectory, and $\theta$ is electron's `velocity' in the dispersive plane:
$\theta = \frac{\partial x}{\partial z}$; similary, y and $\phi$ are displacement
and `velocity' in the y-z plane, $\hat{y}$ is oriented such that $\hat{x}$, 
$\hat{y}$, $\hat{z}$ are orthogonal to each other and and they form a 
right-handed coordinate $\hat{z} = \hat{x} \times \hat{y}$.
$\delta = \frac{\Delta p}{p}$ represents the fraction deviation of momentum
from that of the reference trajectory. To first order, the transportation

One clever method is the usage of sieve slit collimator.

%%%%%%%%%%%%%%%%%%%%%%%%%%%%%%%%%%%%%%%%%%%%%%%%
\subsection{Simulation}
\begin{itemize}
    \item Beam Position
\end{itemize}

%%%%%%%%%%%%%%%%%%%%%%%%%%%%%%%%%%%%%%%%%%%%%%%%
\subsection{Result}


% \subsection{tgt variables}
