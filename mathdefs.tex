% This file defines a number of new commands and operators for nicer and more consistent math typesetting.
% Most are from http://www.tug.org/TUGboat/Articles/tb18-1/tb54becc.pdf

% similar-greater, similar-less
\def\simge{%
    \mathrel{\rlap{\raise 0.511ex
    \hbox{$>$}}{\lower 0.511ex \hbox{$\sim$}}}}
\def\simle{%
    \mathrel{\rlap{\raise 0.511ex
    \hbox{$<$}}{\lower 0.511ex \hbox{$\sim$}}}}

% make vectors bold, rather than using an arrow above
\renewcommand{\vec}[1]{\boldsymbol{#1}}

% Angstrom
\DeclareMathAccent{\ring}{\mathalpha}{operators}{"17}
\providecommand*{\angs}{\ensuremath{\smash{\mathrm{\ring A}}}}

% Ohm
\providecommand*{\ohm}{\ensuremath{\mathrm{\Omega}}}

% micro
\providecommand*{\micro}{\ensuremath{\mu}}

% micrometers
\providecommand*{\um}{\micro m}

% microseconds
\providecommand*{\us}{\micro s}

% Units
\providecommand*{\unit}[1]{\ensuremath{\mathrm{\,#1}}}

% The number `e'
\providecommand*{\eu}{\ensuremath{\mathrm{e}}}

% The imaginary unit
\providecommand*{\iu}{\ensuremath{\mathrm{i}}}

% text on exponent (superscript) level
\providecommand*{\apx}[1]{\ensuremath{^\mathrm{#1}}}

% text on index (subscript) level
\providecommand*{\ped}[1]{\ensuremath{_\mathrm{#1}}}

% degrees (for angles)
\providecommand*{\degree}{\ensuremath{^\circ}}

% degrees celsius
\providecommand*{\celsius}{\ensuremath{\mathrm{^\circ C}}}

% Real and Imaginary parts
\providecommand{\newoperator}[3]{\newcommand*{#1}{\mathop{#2}#3}}
\providecommand{\renewoperator}[3]{\renewcommand*{#1}{\mathop{#2}#3}}
\renewoperator{\Re}{\mathrm{Re}}{\nolimits}
\renewoperator{\Im}{\mathrm{Im}}{\nolimits}

% differential operator
\makeatletter
\providecommand*{\diff}{\@ifnextchar^{\DIfF}{\DIfF^{}}}
\def\DIfF^#1{\mathop{\mathrm{\mathstrut d}}\nolimits^{#1}\gobblespace}
\def\gobblespace{\futurelet\diffarg\opspace}
\def\opspace{%
    \let\DiffSpace\!%
    \ifx\diffarg(%
        \let\DiffSpace\relax
    \else
        \ifx\diffarg[%
            \let\DiffSpace\relax
        \else
            \ifx\diffarg\{%
                \let\DiffSpace\relax
            \fi\fi\fi\DiffSpace}

% total and partial derivatives
%   1st argument [] (optional): derivative order
%   2nd argument {}: function being derived
%   3rd argument {}: derivation variable
\providecommand*{\deriv}[3][]{\frac{\diff^{#1}#2}{\diff #3^{#1}}}
\providecommand*{\pderiv}[3][]{\frac{\partial^{#1}#2}{\partial
#3^{#1}}}
